% Copyright 2009-2011,2015-2018 Dominik Wagenfuehr <dominik.wagenfuehr@deesaster.org>
% Dieses Dokument unterliegt der Creative-Commons-Lizenz
% "Namensnennung-Weitergabe unter gleichen Bedingungen 4.0 International"
% [http://creativecommons.org/licenses/by-sa/4.0/deed.de].

% Pakete
\usepackage{ifxetex,ifluatex}
% Herausfinden, ob XeTeX oder LuaTeX benutzt wird. Abhaengig von der
% Engine muessen andere Schriftpakete geladen werden
\ifluatex
  \usepackage{fontspec}          % UTF8-Kodierung
  \setmainfont{TeX Gyre Pagella} % Pagella-Schrift
\else\ifxetex
  \usepackage{fontspec}          % UTF8-Kodierung
  \setmainfont{TeX Gyre Pagella} % Pagella-Schrift
\else
  \usepackage[utf8]{inputenc}   % UTF8-Kodierung für Umlaute
  \usepackage[T1]{fontenc}      % use TeX encoding then Type 1.
  \usepackage{tgpagella}        % Pagella-Schrift
\fi\fi
\usepackage{ngerman}          % deutsche Silbentrennung
\usepackage{scrlayer-scrpage} % Seitenlayout festlegen
\usepackage{graphicx}         % Einbindung von Grafiken
\usepackage{textcomp}         % Euro-Zeichen und andere
\usepackage{pdfpages}         % PDF-Seiten einbinden
\usepackage{xifthen}          % Wenn-dann-Abfragen
\usepackage{array}            % Kommandos in Tabellenspalten
\usepackage{ulem}             % Unterstreichen
\usepackage{setspace}         % Zeilenabstand festlegen
\usepackage{hyperref}         % Hyperlinks und interne PDF-Verweise
\hypersetup{%
colorlinks=true,
urlcolor=black,
linkcolor=black
}%

% Layout
\hbadness=10000                % unterdrückt unwichtige Fehlermeldungen
\clubpenalty = 10000           % Keine "Schusterjungen"
\widowpenalty = 10000          % Keine "Hurenkinder"
\displaywidowpenalty = 10000

% Linie im Kopf ausblenden und im Fuß eine feine Linie
\KOMAoptions{headsepline=0pt,footsepline=0.4pt}

% Eigene Fusszeile festlegen, Kopfzeile leer
\pagestyle{scrheadings}
\cofoot{\DefVollerName, \DefAbsenderStrasse, \DefAbsenderPLZOrt, \DefTelefon}
\setkomafont{pagefoot}{\upshape\vspace*{0.2\baselineskip}}

% Neue Schrift setzen.
\newcommand{\SetzeSchrift}[2]{%
\ifluatex
  \setmainfont{#2}
\else\ifxetex
  \setmainfont{#2}
\else
  \usepackage{#1}
\fi\fi
}

% Definition des heutigen Datums
\newcommand*{\HeutigerTag}{\today}

% Definitionen zur eigenen Person
\newcommand*{\DefVollerName}{}
\newcommand*{\DefAbsenderStrasse}{}
\newcommand*{\DefAbsenderPLZOrt}{}
\newcommand*{\DefTelefon}{}
\newcommand*{\DefEMail}{}
\newcommand*{\DefOrtDatum}{}
\newcommand*{\DefGeburtstag}{}
\newcommand*{\DefGeburtsort}{}
\newcommand*{\DefDetails}{}
\newcommand*{\DefAusbildungsgrad}{}
\newcommand*{\DefUnterschriftenDatei}{}
\newcommand*{\DefBewerberFoto}{}
\newcommand*{\DefBewerberFotoParam}{}

% Definition zum Ueberschreiben der Daten zur eigenen Person
\newcommand*{\VollerName}[1]{\renewcommand*{\DefVollerName}{#1}}
\newcommand*{\AbsenderStrasse}[1]{\renewcommand*{\DefAbsenderStrasse}{#1}}
\newcommand*{\AbsenderPLZOrt}[1]{\renewcommand*{\DefAbsenderPLZOrt}{#1}}
\newcommand*{\Telefon}[1]{\renewcommand*{\DefTelefon}{#1}}
\newcommand*{\EMail}[2][black]{%
\hypersetup{urlcolor=#1}%
\renewcommand*{\DefEMail}{#2}%
}
\newcommand*{\OrtDatum}[1]{\renewcommand*{\DefOrtDatum}{#1}}
\newcommand*{\Geburtstag}[1]{\renewcommand*{\DefGeburtstag}{#1}}
\newcommand*{\Geburtsort}[1]{\renewcommand*{\DefGeburtsort}{#1}}
\newcommand*{\Details}[1]{\renewcommand*{\DefDetails}{#1}}
\newcommand*{\Ausbildungsgrad}[1]{\renewcommand*{\DefAusbildungsgrad}{#1}}
\newcommand*{\UnterschriftenDatei}[1]{\renewcommand*{\DefUnterschriftenDatei}{#1}}
\newcommand*{\BewerberFoto}[2][6cm]{%
\renewcommand*{\DefBewerberFotoParam}{height=#1}%
\renewcommand*{\DefBewerberFoto}{#2}%
}

% Absender
\newcommand{\Absender}[1][\normalsize]{%
    \rightline{\textbf{#1\DefVollerName}}
    \rightline{\DefAbsenderStrasse}
    \rightline{\DefAbsenderPLZOrt}
    \rightline{\DefTelefon}
    \rightline{\href{mailto:\DefEMail}{\DefEMail}}
}

% Definitionen zum Empfaenger
\newcommand*{\DefFirma}{}
\newcommand*{\DefAbteilung}{}
\newcommand*{\DefAdressatTitel}{}
\newcommand*{\DefAdressatTitelKurz}{}
\newcommand*{\DefAdressatVorname}{}
\newcommand*{\DefAdressatNachname}{}
\newcommand*{\DefAnschriftStrasse}{}
\newcommand*{\DefAnschriftPLZOrt}{}
\newcommand*{\DefAnredeKopf}{}
\newcommand*{\DefAnredeText}{ Damen und Herren}
\newcommand*{\DefZweiterAdressatTitel}{}
\newcommand*{\DefZweiterAdressatTitelKurz}{}
\newcommand*{\DefZweiterAdressatVorname}{}
\newcommand*{\DefZweiterAdressatNachname}{}
\newcommand*{\DefZweiterAdressatAnredeKopf}{}
\newcommand*{\DefZweiterAdressatAnredeText}{}

% Definitionen zum Ueberschreiben der Daten des Empfaengers
\newcommand*{\Firma}[1]{\renewcommand*{\DefFirma}{#1}}
\newcommand*{\Abteilung}[1]{\renewcommand*{\DefAbteilung}{#1}}
\newcommand*{\AdressatTitel}[2][]{%
    \ifthenelse{\equal{#2}{}}{%
        \renewcommand*{\DefAdressatTitel}{}%
    }{%
        \renewcommand*{\DefAdressatTitel}{#2 }%
    }%
    \ifthenelse{\equal{#1}{}}{%
        \renewcommand*{\DefAdressatTitelKurz}{\DefAdressatTitel}%
    }{%
        \renewcommand*{\DefAdressatTitelKurz}{#1 }%
    }%
}
\newcommand*{\AdressatVorname}[1]{\renewcommand*{\DefAdressatVorname}{#1}}
\newcommand*{\AdressatNachname}[1]{\renewcommand*{\DefAdressatNachname}{#1}}
\newcommand*{\AnschriftStrasse}[1]{\renewcommand*{\DefAnschriftStrasse}{#1}}
\newcommand*{\AnschriftPLZOrt}[1]{\renewcommand*{\DefAnschriftPLZOrt}{#1}}
\newcommand*{\Anrede}[1]{%
    \ifthenelse{\equal{\DefAdressatNachname}{}}{%
        \renewcommand*{\DefAnredeKopf}{}%
        \renewcommand*{\DefAnredeText}{ Damen und Herren}%
    }{%
        \ifthenelse{\equal{#1}{Herr}}{%
            \renewcommand*{\DefAnredeKopf}{Herrn \DefAdressatTitelKurz}%
            \renewcommand*{\DefAnredeText}{r Herr \DefAdressatTitel}%
        }{%
            \ifthenelse{\equal{#1}{Frau}}{%
                \renewcommand*{\DefAnredeKopf}{Frau \DefAdressatTitelKurz{}}%
                \renewcommand*{\DefAnredeText}{ Frau \DefAdressatTitel{}}%
            }{
                \errmsg{Anrede darf sein: '','Herr','Frau'.}
            }%
        }%
    }%
}
\newcommand*{\ZweiterAdressatTitel}[2][]{%
    \ifthenelse{\equal{#1}{}}{%
        \renewcommand*{\DefZweiterAdressatTitel}{}%
    }{%
        \renewcommand*{\DefZweiterAdressatTitel}{#2 }%
    }%
    \ifthenelse{\equal{#1}{}}{%
        \renewcommand*{\DefZweiterAdressatTitelKurz}{\DefZweiterAdressatTitel}%
    }{%
        \renewcommand*{\DefZweiterAdressatTitelKurz}{#1 }%
    }%
}
\newcommand*{\ZweiterAdressatVorname}[1]{\renewcommand*{\DefZweiterAdressatVorname}{#1}}
\newcommand*{\ZweiterAdressatNachname}[1]{\renewcommand*{\DefZweiterAdressatNachname}{#1}}
\newcommand*{\ZweiterAdressatAnrede}[1]{%
    \ifthenelse{\equal{\DefZweiterAdressatNachname}{}}{%
        \renewcommand*{\DefZweiterAdressatAnredeKopf}{}%
        \renewcommand*{\DefZweiterAdressatAnredeText}{}%
    }{%
        \ifthenelse{\equal{#1}{Herr}}{%
            \renewcommand*{\DefZweiterAdressatAnredeKopf}{Herrn \DefZweiterAdressatTitelKurz}%
            \renewcommand*{\DefZweiterAdressatAnredeText}{r Herr \DefZweiterAdressatTitel}%
        }{%
            \ifthenelse{\equal{#1}{Frau}}{%
                \renewcommand*{\DefZweiterAdressatAnredeKopf}{Frau \DefZweiterAdressatTitelKurz}%
                \renewcommand*{\DefZweiterAdressatAnredeText}{ Frau \DefZweiterAdressatTitel}%
            }{
                \errmsg{Anrede darf sein: '','Herr','Frau'.}
            }%
        }%
    }%
}

% Adressat
\newcommand{\Anschrift}{%
    \textbf{\DefFirma}\\
    \ifthenelse{\equal{\DefAnredeKopf}{}}{}{%
        \DefAnredeKopf{}%
        \ifthenelse{\equal{\DefAdressatVorname}{}}{}{%
            \DefAdressatVorname{} %
        }%
        \ifthenelse{\equal{\DefAdressatNachname}{}}{}{%
            \DefAdressatNachname{} \\
        }%
    }%
    \ifthenelse{\equal{\DefZweiterAdressatAnredeKopf}{}}{}{%
        \DefZweiterAdressatAnredeKopf{}%
        \ifthenelse{\equal{\DefZweiterAdressatVorname}{}}{}{%
            \DefZweiterAdressatVorname{} %
        }%
        \ifthenelse{\equal{\DefZweiterAdressatNachname}{}}{}{%
            \DefZweiterAdressatNachname{} \\
        }%
    }%
    \ifthenelse{\equal{\DefAbteilung}{}}{}{%
        \DefAbteilung\\
    }%
    \DefAnschriftStrasse\\
    \DefAnschriftPLZOrt\\
}

% Makro für Ort und Unterschrift
\newcommand*{\UnterschriftOrt}{%
    \Unterschrift~\\
    \DefOrtDatum
}

% Unterschrift als Bilddatei einfügen
\newcommand*{\Unterschrift}{%
    \ifthenelse{\equal{\DefUnterschriftenDatei}{}}{%
        \vspace{2\baselineskip}%
    }{%
        \includegraphics{\DefUnterschriftenDatei}%
    }
}

% Ausrichtung der Ueberschrift
\newcommand*{\DefUeberschriftAusrichtung}{links}%
\newcommand*{\UeberschriftAusrichtung}[1]{%
    \ifthenelse{\equal{#1}{links}}{%
        \renewcommand*{\DefUeberschriftAusrichtung}{#1}%
    }{%
        \ifthenelse{\equal{#1}{rechts}}{%
            \renewcommand*{\DefUeberschriftAusrichtung}{#1}%
        }{%
            \ifthenelse{\equal{#1}{mittig}}{%
                \renewcommand*{\DefUeberschriftAusrichtung}{#1}%
            }{%
                \errmsg{Ausrichtung darf sein: 'links','rechts','mittig'.}
            }%
        }%
    }%
}

% Schriftgroesse der Ueberschrift
\newcommand*{\DefUeberschriftGroesse}{\LARGE}%
\newcommand*{\UeberschriftGroesse}[1]{\renewcommand*{\DefUeberschriftGroesse}{#1}}

% Interne Umgebung fuer die Darstellung der Ueberschrift
\newenvironment*{AbschnittIntern}{%
    \ifthenelse{\equal{\DefUeberschriftAusrichtung}{mittig}}{%
        \begin{center}%
    }{%
        \ifthenelse{\equal{\DefUeberschriftAusrichtung}{links}}{%
            \hspace*{\fill}%
        }{%
        }%
    }%
    \normalfont\DefUeberschriftGroesse\scshape%
}{%
    \\%
    \ifthenelse{\equal{\DefUeberschriftAusrichtung}{mittig}}{%
        \end{center}%
    }{%
    }%
}

% Abschnitt für die Hauptkapitel
\newcommand*{\Abschnitt}[1]{%
    \phantomsection%
    \begin{AbschnittIntern}
    #1%
    \end{AbschnittIntern}
    \addcontentsline{toc}{section}{#1}%
}

% Unterabschnitte im Lebenslauf und Anlagenverzeichnis
\newcommand*{\Unterabschnitt}[1]{%
\phantomsection%
\ifthenelse{\boolean{UnterabschnittBuendig}}{%
    \hspace{\LenEinschubCV}%
}{}%
{\normalfont\Large\bfseries\scshape #1}%
\vskip-\baselineskip\hrulefill\\[-0.5\baselineskip]%
\addcontentsline{toc}{subsection}{#1}%
}

% Ueberschrift im Lebenslauf festlegen.
% Definitionen zum Empfaenger
\newcommand*{\DefUeberschriftLebenslauf}{Lebenslauf}
\newcommand*{\UeberschriftLebenslauf}[1]{\renewcommand*{\DefUeberschriftLebenslauf}{#1}}

% Lebenslauf
\newenvironment{Lebenslauf}
{%
    \Abschnitt{\DefUeberschriftLebenslauf}
}{%
    % Seite um eine Zeile vergrößern, damit Unterschrift besser draufpasst.
    \enlargethispage{2\baselineskip}%
    \UnterschriftOrt%
    \clearpage
}

% Seitenumbruch innerhalb des Lebenslaufs
\newcommand*{\NeueSeiteAbschnittCV}{%
\clearpage
\Abschnitt{\DefUeberschriftLebenslauf}
}

% Per Standard gibt es keinen Tabellenabstand.
\setlength{\tabcolsep}{0pt}%

% Legt den Einschub des Inhalts eines CV-Abschnitte (d.h. der Tabelle) fest.
\newboolean{UnterabschnittBuendig}%
\setboolean{UnterabschnittBuendig}{false}%
\newlength{\LenEinschubCV}
\setlength{\LenEinschubCV}{8pt}
\newcommand*{\EinschubCV}[2][]{%
\setlength{\LenEinschubCV}{#2}%
\ifthenelse{\equal{#1}{}}{%
    \setboolean{UnterabschnittBuendig}{false}%
}{%
    \setboolean{UnterabschnittBuendig}{true}%
}%
}

% Tabelle für Lebenslauf
\newlength{\AbstandAbschnitt}
\newenvironment{AbschnittCV}[2][0.5\baselineskip]
{%
    \Unterabschnitt{#2}%
    
    \setlength{\AbstandAbschnitt}{#1}%
}{%
    \vspace{\AbstandAbschnitt}%
}

% Eintrag im Lebenslauf
\newcommand{\EintragCV}[2]
{%
    \hspace*{\LenEinschubCV}%
    \begin{tabular}{p{0.3\linewidth}p{0.7\linewidth-\LenEinschubCV}}%
    #1 & #2%
    \end{tabular}\\[0.5\baselineskip]%
}

% Unterabschnitt in Lebenslauftabelle mit sonstigen Themen
% Achtung, keine Umgebung, sondern ein Kommando!
% Das optionale Argument wird nicht mehr genutzt!
\newlength{\AbstandUnterabschnitt}
\newcommand{\UnterabschnittCV}[3][0\baselineskip]
{%
    \textit{#2}:
    \begin{itemize}
    #3
    \end{itemize}
}

% Anrede im Brief
\newcommand*{\DefAnrede}{%
    Sehr geehrte\DefAnredeText{}%
    \ifthenelse{\equal{\DefAdressatNachname}{}}{}{%
        \DefAdressatNachname{}%
    }%
    ,%
    \ifthenelse{\equal{\DefZweiterAdressatNachname}{}}{}{
        \\%
        sehr geehrte\DefZweiterAdressatAnredeText{}%
        \DefZweiterAdressatNachname{}%
        ,%
    }%
}

% Die Grußformel am Anschreibenende
\newcommand{\Grussworte}{%
Über die Einladung zu einem persönlichen Gespräch freue ich mich und verbleibe

mit freundlichen Grüßen

\Unterschrift~\\
}

% Definition der Bewerberstelle
\newcommand*{\DefBewerberstelle}{}

% Definitionen zum Ueberschreiben der Bewerberstelle
\newcommand*{\Bewerberstelle}[1]{\renewcommand*{\DefBewerberstelle}{#1}}

% Betreff
\newcommand*{\Betreff}{\textbf{Bewerbung als \DefBewerberstelle}}

% Etwas Abstand vor dem Ort, um die Seite besser aufteilen zu können.
\newlength{\LenAbstandVorAnschreiben}
\setlength{\LenAbstandVorAnschreiben}{\baselineskip}
\newcommand*{\AbstandVorAnschreiben}[1]{\setlength{\LenAbstandVorAnschreiben}{#1\baselineskip}}

% Abstand zwischen den Adressen im Anschreiben.
\newlength{\LenAbstandZwischenAdressen}
\setlength{\LenAbstandZwischenAdressen}{0\baselineskip}
\newcommand*{\AbstandZwischenAdressen}[1]{\setlength{\LenAbstandZwischenAdressen}{#1\baselineskip}}

% Die Anschreibeseite wird bei Bedarf etwas vergrößert, da es hier keine
% Fußzeile gibt.
\newlength{\LenSeiteVergroessern}
\setlength{\LenSeiteVergroessern}{0\baselineskip}
\newcommand*{\AnschreibenSeiteVergroessern}[1]{\setlength{\LenSeiteVergroessern}{#1\baselineskip}}

\newcommand{\AnschreibenKopf}
{%
    % Anschreiben ohne Fuss- oder Kopfzeile
    \thispagestyle{empty}

    % Seite ggf. etwas vergroessern
    \enlargethispage{\LenSeiteVergroessern}
    
    \begin{spacing}{1.05}
        % Meine Anschrift
        \Absender

        % ggf. will man etwas Abstand zwischen den Adressen.
        \vspace*{\LenAbstandZwischenAdressen}

        % Anschrift der Firma
        \Anschrift

        % ggf. braucht man hier etwas Abstand, je nachdem, wie
        % viel Text man im Anschreiben hat.
        \vspace*{\LenAbstandVorAnschreiben}

        % Ort und Datum
        \rightline{\DefOrtDatum}
    \end{spacing}

}

% Abstand zwischen Unterschrift und den Anlagen im Anschreiben
\newlength{\LenAbstandVorAnlagen}
\setlength{\LenAbstandVorAnlagen}{\baselineskip}
\newcommand*{\AbstandVorAnlagen}[1]{\setlength{\LenAbstandVorAnlagen}{#1\baselineskip}}

% Definition des Textes im Anschreiben
\newcommand{\DefAnschreibenAnlage}{}

% Definition zum Setzen des Textes im Anschreiben
\newcommand{\AnschreibenAnlage}[1]{\renewcommand{\DefAnschreibenAnlage}{#1}}

\newenvironment{Anschreiben}
{%
    % Titel und Autor des PDFs werden automatisch festgelegt,
    % was aber erst nach der Definition des Namens geht.
    \hypersetup{%
        pdftitle={Bewerbung bei \DefFirma},
        pdfauthor={\DefVollerName},
        pdfcreator={\DefVollerName}
    }

    \AnschreibenKopf
    \begin{spacing}{1.15}
        \Betreff
        \begin{flushleft}
            \DefAnrede

}{%

            \Grussworte
            \vspace{\LenAbstandVorAnlagen}
            \DefAnschreibenAnlage
        \end{flushleft}
    \end{spacing}
    \clearpage
}

\newcommand{\MeineSeite}
{%
    \Abschnitt{Zu meiner Person}

    \begin{spacing}{1.3}
        \Absender[\large]
        \vspace{1.5\baselineskip}

        \rightline{geboren am \DefGeburtstag}
        \rightline{in \DefGeburtsort}
        \vspace{0.5\baselineskip}

        \rightline{\DefDetails}

        \rule{10cm}{0.6pt} \\
        \ifthenelse{\equal{\DefAusbildungsgrad}{}}{}{%
            {\large \textbf{Ausbildungsgrad}}\\
            \DefAusbildungsgrad\\
        }%

        \ifthenelse{\equal{\DefBewerberFoto}{}}{}{%
            \expandafter\includegraphics\expandafter[\DefBewerberFotoParam]{\DefBewerberFoto}\\
        }%

    \end{spacing}
    \clearpage
}

\newenvironment{Motivation}
{%
    \Abschnitt{Über mich und meine Motivation}
 
    \begin{spacing}{1.3}
        \begin{flushleft}

}{%

        \end{flushleft}
    \end{spacing}
    
    \UnterschriftOrt
    \clearpage
}

% Definition der Abschnitte
\newcommand*{\AbschnittAnlage}[1]{%
\vspace{\baselineskip}%
\Unterabschnitt{#1}%
}

% Definition der Auflistung
\newenvironment{Auflistung}
{\begin{itemize}}
{\end{itemize}}

\newcommand{\punkt}{\item}

% Definition des Anlagenverzeichnisses
\newenvironment{Anlagenverzeichnis}{%
    \Abschnitt{Anlagenverzeichnis}

    \begin{spacing}{1.1}
}{%
    \end{spacing}
    \clearpage
}

% Eine einzelne Anlage
\newcommand*{\Anlage}[2][]{%
\ifthenelse{\equal{#1}{}}{%
\item #2
}{%
\item \hyperlink{#1.1}{\uline{#2}}
}%
}

% Anlagen einfuegen
% Option "quer" (oder irgendein anderes Wort) für Querformat.
\newcommand*{\AnlageEinfuegen}[2][]{%
\ifthenelse{\equal{#1}{}}{%
\includepdf[pages=-,link,linkname=#2]{#2}%
}{%
\includepdf[landscape,pages=-,link,linkname=#2]{#2}%
}%
}



